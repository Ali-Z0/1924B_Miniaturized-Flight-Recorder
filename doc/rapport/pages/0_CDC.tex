\onecolumn

\begin{figure}
	\begin{minipage}{0.47\textwidth}
		\centering
		\includegraphics[width=.4\textwidth,left,]{./ETML-ES-LOGO.png}
	\end{minipage}
	
	\hfill
	\begin{minipage}{0.7\textwidth}
		\raggedleft
		\LARGE \textbf{Boîte noire miniaturisée\\ 2023, 1942B}
	\end{minipage}
\end{figure}

\begin{figure}
	
	\hfill
	
	
	\begin{minipage}{1\textwidth}
		% ---- DESCRIPTION ----
		\section{Cahier des charges}
		
		\subsection{Description}
		L'objectif de ce projet est de stocker des données de mesures et de localisation d'un avion par une centrale inertielle et un GPS / GNSS, afin d'enregistrer des informations quant aux caractéristiques du vol et à sa trajectoire. Cela permettrai par exemple, en cas de crash, de déduire les éventuelles causes. 
	\end{minipage} \vspace{+4mm}
	
	\begin{minipage}{1\textwidth}
		
		\subsection{Aperçu}
		\begin{itemize}
			\item	Sauvegarde d’un set de donnée chaque 10s.
			\item	Résistance aux chocs.
			\item	Autonomie de 2 jours.
			\item	\gls{gps} / \gls{gnss}.
			\item	Timestamp.
			\item	\gls{centrale inertielle}, Sensing sur 9 axes :
			\subitem Accéléromètre 3-axes. 
			\subitem	Gyroscope 3-axes.
			\subitem	Magnétomètre 3-axes. 
			\subitem	Senseur de température
			\item Traceur de G.
			\item Charge de la batterie par connecteur USB.
			\item Lecture des données par connecteur USB.
			\item Lecture des données par radio-fréquence.
		\end{itemize}
		
		
	\end{minipage}
	
\end{figure}


\clearpage

% ---- TACHES ----
\subsection{Tâches à réaliser}
Développement et intégration d’un PCB avec capteurs et logging sur carte SD dans une lampe de plongée étanche.
\begin{itemize}
	\item[•] Développement schématique 
	\subitem- Fonctionnement MCU.
	\subitem-	Périphériques de mesures et de sauvegarde / Bus de communication.
	\subitem-	Gestion batterie 
	\item[•]	Routage pour intégration dans boitier de lampe de plongée 200x45mm.
	\item[•]	Programmation mesure et sauvegarde chaque 100ms.
	\subitem-	Configuration MCU.
	\subitem-	Configuration des périphériques de mesure pour 9-DOF.
	\subitem-	Configuration des périphériques de sauvegarde (Carte SD).
	\subitem-	Configuration et communication avec l'interface.
	\subitem-	Communication et traitement des données mesurées.
\end{itemize}


% ---- SChema de principe ----


% ---- Description des blocs ----
\subsection{Description des blocs}

\clearpage

% ---- JALONS ----
\subsection{Jalons principaux}


\subsection{Livrable}
\begin{itemize}
	\item[•] Les fichiers sources de CAO électronique des PCB réalisés
	\item[•] Tout le nécessaire à fabriquer un exemplaire hardware de chaque :
	\item[•] fichiers de fabrication (GERBER) / liste de pièces avec références pour commande / implantation
	\item[•] Prototype fonctionnel
	\item[•] Modifications / dessins mécaniques, etc
	\item[•] Les fichiers sources de programmation microcontrôleur (.c  / .h)
	\item[•] Tout le nécessaire pour programmer les microcontrôleurs (logiciel ou fichier .hex)
	\item[•] Un calcul / estimation des coûts
	\item[•] Un rapport contenant les calculs - dimensionnement de composants - structogramme, etc.
\end{itemize}

\clearpage
