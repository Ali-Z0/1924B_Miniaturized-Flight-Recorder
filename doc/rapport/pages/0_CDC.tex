\onecolumn

\begin{figure}
	\begin{minipage}{0.47\textwidth}
		\centering
		\includegraphics[width=.4\textwidth,left,]{./ETML-ES-LOGO.png}
	\end{minipage}
	\begin{minipage}{0.7\textwidth}
		%\raggedleft
		\LARGE \textbf{Boîte noire miniaturisée\\ 2023, 1942B}
	\end{minipage}
\end{figure}


% ---- DESCRIPTION ----
\section{Introduction}
Les enregistreurs de données de vol, communément appelés "boîtes noires", sont essentiels dans l'aviation contemporaine. Ils jouent un rôle crucial dans la sécurité aérienne et la compréhension des phénomènes aéronautiques en capturant de manière inaltérable des informations vitales. Ces données sont précieuses pour l'investigation d'incidents ou d'accidents aériens, permettant une reconstitution précise des événements à bord des aéronefs. Les boîtes noires sont ainsi des outils indispensables pour détecter des anomalies, des dysfonctionnements, voire des défaillances techniques, contribuant ainsi à renforcer la sécurité dans le secteur aéronautique.



Une boîte noire comprend deux principaux enregistreurs conçus pour résister aux conditions extrêmes : l'enregistreur de la voix du cockpit (CVR), qui enregistre les communications du cockpit, et l'enregistreur de données de vol (FDR), chargé de collecter divers paramètres techniques de l'aéronef. Ces paramètres englobent la vitesse de l'aéronef, son altitude, son orientation, les données des moteurs, les paramètres aérodynamiques, les données de navigation, les commandes du pilote, les transmissions radio, les données environnementales, et bien d'autres éléments essentiels.



Dans ce cadre, le présent projet a pour objectif la collecte et le stockage des données de mesures et de localisation d'un aéronef au moyen d'une centrale inertielle et d'un système de positionnement GPS/GNSS. Par le biais de la conjonction de ces technologies, il est envisageable d'enregistrer des données d'une précision remarquable concernant les paramètres du vol et la trajectoire empruntée par l'aéronef. En cas de survenue d'un incident, ces enregistrements jouent un rôle déterminant en permettant d'établir les causes potentielles. Pour une analyse approfondie des spécifications techniques, veuillez consulter le cahier des charges complet disponible en annexe.

\subsection{Aperçu}
\begin{itemize}
	\item[•]	Sauvegarde des données inertielles chaque 500ms par défaut.
	\item[•]	Sauvegarde des données de localisation chaque 5'000ms par défaut.
	\item[•]	Possibilité de configurer les temps de sauvegarde.
	\item[•]	Résistance aux chocs.
	\item[•]	Bonne autonomie / Low power.
	\item[•]	\Gls{gps}
	\item[•]	\Gls{gnss}.
	\item[•]	\Gls{timestamp} par satellite.
	\item[•]	\gls{centrale inertielle} :
	\subitem- 	Accéléromètre 3-axes. 
	\subitem-	Gyroscope 3-axes.
	\item[•] Charge, lecture et config. par USB-C.
\end{itemize}

% ---- TACHES ----
\subsection{Tâches à réaliser}
Développement et intégration d’un PCB avec capteurs et logging sur carte SD dans un boitier compact.
\begin{itemize}
	\item[•] Développement schématique 
	\subitem- Fonctionnement MCU.
	\subitem-	Périphériques de mesures et de sauvegarde / Bus de communication.
	\subitem-	Gestion batterie 
	\item[•]	Routage pour intégration dans boitier résistant aux chocs.
	\item[•]	Programmation mesure et sauvegarde des données.
	\subitem-	Configuration MCU.
	\subitem-	Configuration du périphérique de mesure pour \gls{imu}.
	\subitem-	Configuration du périphérique de sauvegarde (Carte SD).
	\subitem-	Configuration du périphérique de localisation \gls{gps}/\gls{gnss}.
	\subitem-	Configuration et communication avec l'interface.
	\subitem-	Communication et traitement des données mesurées.
\end{itemize}

\subsection{Schéma de principe}
% ---- SChema de principe ----
\begin{figure}[h]
	\centering
	\includegraphics[width=0.4\linewidth]{../figures/cdc/schema_principe}
	\caption{Schéma de principe.}
	\source{Auteur}
	\label{fig:schemaprincipe}
\end{figure}

Ce système électronique de mini boîte noire pour avion, serait capable d'enregistrer des informations récentes sur les données inertielle et la position d'un vol, dans une mémoire non volatile (carte microSD). Le dispositif, abrité dans un boîtier plastique pour assurer une réception optimale des données \gls{gps} et une installation compact. Celui-ci fournirait des données via un port USB pour l'extraction des mesures sur la carte microSD ou pour configurer des paramètres tels que les intervalles de mesures. Les mesures sur les trois axes d'accélération et de vitesse angulaire seraient recueillies par défaut toutes les 500 ms et les données de position \gls{gps} toutes les 5000 ms. Des intervalles d'enregistrement plus longs sont envisageables pour optimiser la durée de vie de la carte SD, selon la taille et l'organisation des données. Le dispositif sauvegarderait les données des 15 dernières minutes de vol (ou plus) dans un fichier CSV pour traitement ultérieur, selon le principe FIFO. L'objectif principal du prototype est de privilégier la compacité pour minimiser son encombrement à bord de l'avion.


\clearpage

% ---- JALONS ----
\subsection{Planification}
\begin{figure}[h]
	\centering
	\includegraphics[width=1\linewidth]{../figures/cdc/planif}
	\caption{Planification, Diagramme de gantt.}
	\source{Auteur}
	\label{fig:planification}
\end{figure}
\begin{figure}[h]
	\centering
	\includegraphics[width=1\linewidth]{../figures/cdc/planif_theorique}
	\caption{Planification théorique.}
	\source{Auteur}
	\label{fig:planiftheorique}
\end{figure}


\subsection{Livrable}
\begin{itemize}
	\item[•] Les fichiers sources de CAO électronique des PCB réalisés
	\item[•] Tout le nécessaire à fabriquer un exemplaire hardware de chaque :
	\item[•] fichiers de fabrication (GERBER) / liste de pièces avec références pour commande / implantation
	\item[•] Prototype fonctionnel
	\item[•] Modifications / dessins mécaniques, etc
	\item[•] Les fichiers sources de programmation microcontrôleur (.c  / .h)
	\item[•] Tout le nécessaire pour programmer les microcontrôleurs (logiciel ou fichier .hex)
	\item[•] Un calcul / estimation des coûts
	\item[•] Un rapport contenant les calculs - dimensionnement de composants - structogramme, etc.
\end{itemize}

\clearpage
