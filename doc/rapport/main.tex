% !TeX root = main.tex
% !TeX spellcheck = fr_FR
\documentclass[12pt,a4paper,twoside]{article}
\usepackage[scaled]{helvet}
% Packages and macros
\usepackage[T1]{fontenc}
\usepackage{graphicx}
\usepackage{mathptmx}
\usepackage{pgf}
\usepackage{pgfpages}
\usepackage{booktabs}
\usepackage{fancyhdr}
\usepackage{datetime}
\usepackage{enumerate}
\usepackage{pifont}
\usepackage{amssymb}
\usepackage[export]{adjustbox}
\usepackage[margin=1in]{geometry}
\usepackage[french]{babel}
\usepackage{caption}
\usepackage{tikz}
\usepackage{tabularx}
\usepackage{gensymb}
\usepackage{pdfpages}
\usepackage{caption}
\usepackage{subcaption}
\usepackage{float}
%\usepackage{listings}
\usepackage{afterpage}
\usepackage[bottom]{footmisc}
\usepackage{textcomp}
\usepackage{fontawesome5}
\usepackage[outdir=./]{epstopdf}

\usepackage{amsmath}
% ---- CODE Packages ---
\usepackage[most]{tcolorbox}
\tcbset {
	base/.style={
		arc=0mm, 
		bottomtitle=0.5mm,
		boxrule=0mm,
		colbacktitle=black!10!white, 
		coltitle=black, 
		fonttitle=\bfseries, 
		left=2.5mm,
		leftrule=1mm,
		right=3.5mm,
		title={#1},
		toptitle=0.75mm, 
	}
}

\definecolor{brandblue}{rgb}{0.34, 0.7, 1}
\newtcolorbox{mainbox}[1]{
	colframe=brandblue, 
	base={#1}
}

\newtcolorbox{subbox}[1]{
	colframe=black!30!white,
	base={#1}
}
\usepackage{etoolbox}
\usepackage[newfloat]{minted}
\BeforeBeginEnvironment{minted}{\begin{mainbox} }%
\AfterEndEnvironment{minted}{\end{mainbox}}%
\newenvironment{code}{\captionsetup{type=listing}}{}
\SetupFloatingEnvironment{listing}{name=Code Source}
% ----------------------


\usepackage{hyperref}
\hypersetup{
	colorlinks=true,
	linkcolor=blue,
	filecolor=magenta,      
	urlcolor=cyan,
	pdftitle={Overleaf Example},
	pdfpagemode=FullScreen,
}

\usepackage[sort=none, abbreviations]{glossaries-extra}

\newglossaryentry{centrale inertielle}
{
	name=Centrale inertielle,
	description={Instrument utilisé en navigation, capable d'intégrer les mouvements d'un mobile (accélération et vitesse angulaire) pour estimer son orientation (angles de roulis, de tangage et de cap), sa vitesse linéaire et sa position.}
}
\newglossaryentry{timestamp}
{
	name=Timestamp,
	description={Enrengistrement de l'heure et/ou la date d'un événement.}
}

\newglossaryentry{GPS}
{
	name=GPS,
	description={Le système de positionnement global (GPS) est un service public américain qui fournit aux utilisateurs des services de positionnement, de navigation et de synchronisation (PNT). Ce système se compose de trois segments : le segment spatial, le segment de contrôle et le segment utilisateur. L'U.S. Space Force développe, entretient et exploite les segments spatial et de contrôle.}
}

\newglossaryentry{GNSS}
{
	name=GNSS,
	description={Le système mondial de navigation par satellite (GNSS) est un terme général décrivant toute constellation de satellites qui fournit des services de positionnement, de navigation et de synchronisation (PNT) à l'échelle mondiale ou régionale.}
}

\newglossaryentry{PIC32}
{
	name=PIC32,
	description={Famille de microcontrôleur 32-bits de Microchip.}
}
\newglossaryentry{FTDI}
{
	name=FTDI,
	description={Composant Future Technology Devices International. Ici sert de convertisseur USB to UART.}
}
\newglossaryentry{harmony}
{
	name=Harmony,
	description={Configurateur graphique / générateur de code pour les microcontrôleurs de Microchip.}
}

\newglossaryentry{datasheet}
{
	name=Datasheet,
	description={Document du fabricant fournissant les spécifications d'un produit.}
}



\newabbreviation{mcu}{MCU}{microcontrôleur.}
\newabbreviation{pcb}{PCB}{circuit imprimé.}
\newabbreviation{imu}{IMU}{centrale inertielle.}
\newabbreviation{rf}{RF}{radio-fréquence.}
\newabbreviation{gps}{GPS}{global Positioning System.}
\newabbreviation{gnss}{GNSS}{global navigation satellite systems.}





\newcommand{\source}[1]{\vspace{-11pt} \caption*{\small \textit{Source: {#1}} }}

\newcommand\blankpage{%
	\null
	\thispagestyle{empty}%
	\addtocounter{page}{-1}%
	\newpage}

\usepackage{verbatim}

\newdateformat{monthyeardate}{%
  \monthname[\THEMONTH], \THEYEAR}
  
% Affichages des paragraphes dans TDM
\setcounter{tocdepth}{4} 
\setcounter{secnumdepth}{4}

\begin{document}
\pagestyle{fancy}
\lhead{Travail de diplôme 1924B}
\chead {\today}
\rhead{Mini boîte noire}

% ------------------------- TITLE PAGE INSERTION ------------------------
\begin{titlepage} % Suppresses displaying the page number on the title page and the subsequent page counts as page 1
	\newcommand{\HRule}{\rule{\linewidth}{0.5mm}} % Defines a new command for horizontal lines, change thickness here
	
	\center % Centre everything on the page
	
	%------------------------------------------------
	%	Headings
	%------------------------------------------------
	
	\textsc{\LARGE Ecole des métiers de Lausanne \\ Ecole supérieure}\\[1.5cm] % Main heading such as the name of your university/college
	
	\textsc{\Large Génie électrique}\\[0.5cm] % Major heading such as course name
	
	\textsc{\large Système d'enregistrement de trajectoires de vol}\\[0.5cm] % Minor heading such as course title
	
	%------------------------------------------------
	%	Title
	%------------------------------------------------
	\vspace{+25mm}
	
	\HRule\\[0.4cm]
	
	{\huge\bfseries Boîte noire miniaturisée}\\[0.4cm] % Title of your document
	
	\HRule\\[1.5cm]
	
	\vspace{+10mm}
	
	%------------------------------------------------
	%	Author(s)
	%------------------------------------------------

	\begin{center}
		\large
		\textit{Auteur}\\
		Ali \textsc{Zoubir} \vspace{+5mm}\\ 
		\large
		\textit{Superviseur} \\
		Juan José \textsc{Moreno} \vspace{+5mm}\\
		\large
		\textit{Mandant}\\
		\textsc{Association pour le Maintien \\ du Patrimoine Aéronautique}
	\end{center}

	
	% If you don't want a supervisor, uncomment the two lines below and comment the code above
	%{\large\textit{Author}}\\
	%John \textsc{Smith} % Your name
	
	%------------------------------------------------
	%	Date
	%------------------------------------------------
	
	\vfill
	\vfill
	%\includegraphics[width=5cm]{LOGO-PROJ}
	% \vfill\vfill\vfill % Position the date 3/4 down the remaining page
	\vfill
	
	{\large\today} % Date, change the \today to a set date if you want to be precise
	
	%------------------------------------------------
	%	Logo
	%------------------------------------------------
	
	%\vfill\vfill
	%\includegraphics[width=0.2\textwidth]{placeholder.jpg}\\[1cm] % Include a department/university logo - this will require the graphicx package
	
	%----------------------------------------------------------------------------------------
	
	\vfill % Push the date up 1/4 of the remaining page
	
\end{titlepage} 
\afterpage{\blankpage}

\clearpage

% --------------------- TABLE OF CONTENTS  ------------------------------- 
\tableofcontents
\clearpage

\printunsrtglossary[type=abbreviations]
\printunsrtglossary % default: style=list, type=main

% ---------- CAHIER DES CHARGES ---------------------------------\\
\onecolumn

\begin{figure}
	\begin{minipage}{0.47\textwidth}
		\centering
		\includegraphics[width=.4\textwidth,left,]{./ETML-ES-LOGO.png}
	\end{minipage}
	\begin{minipage}{0.7\textwidth}
		%\raggedleft
		\LARGE \textbf{Boîte noire miniaturisée\\ 2023, 1942B}
	\end{minipage}
\end{figure}


% ---- DESCRIPTION ----
\section{Introduction}
Les enregistreurs de données de vol, communément appelés "boîtes noires", sont essentiels dans l'aviation contemporaine. Ils jouent un rôle crucial dans la sécurité aérienne et la compréhension des phénomènes aéronautiques en capturant de manière inaltérable des informations vitales. Ces données sont précieuses pour l'investigation d'incidents ou d'accidents aériens, permettant une reconstitution précise des événements à bord des aéronefs. Les boîtes noires sont ainsi des outils indispensables pour détecter des anomalies, des dysfonctionnements, voire des défaillances techniques, contribuant ainsi à renforcer la sécurité dans le secteur aéronautique.



Une boîte noire comprend deux principaux enregistreurs conçus pour résister aux conditions extrêmes : l'enregistreur de la voix du cockpit (CVR), qui enregistre les communications du cockpit, et l'enregistreur de données de vol (FDR), chargé de collecter divers paramètres techniques de l'aéronef. Ces paramètres englobent la vitesse de l'aéronef, son altitude, son orientation, les données des moteurs, les paramètres aérodynamiques, les données de navigation, les commandes du pilote, les transmissions radio, les données environnementales, et bien d'autres éléments essentiels.



Dans ce cadre, le présent projet a pour objectif la collecte et le stockage des données de mesures et de localisation d'un aéronef au moyen d'une centrale inertielle et d'un système de positionnement GPS/GNSS. Par le biais de la conjonction de ces technologies, il est envisageable d'enregistrer des données d'une précision remarquable concernant les paramètres du vol et la trajectoire empruntée par l'aéronef. En cas de survenue d'un incident, ces enregistrements jouent un rôle déterminant en permettant d'établir les causes potentielles. Pour une analyse approfondie des spécifications techniques, veuillez consulter le cahier des charges complet disponible en annexe.

\subsection{Aperçu}
\begin{itemize}
	\item[•]	Sauvegarde des données inertielles chaque 500ms par défaut.
	\item[•]	Sauvegarde des données de localisation chaque 5'000ms par défaut.
	\item[•]	Possibilité de configurer les temps de sauvegarde.
	\item[•]	Résistance aux chocs.
	\item[•]	Bonne autonomie / Low power.
	\item[•]	\Gls{gps}
	\item[•]	\Gls{gnss}.
	\item[•]	\Gls{timestamp} par satellite.
	\item[•]	\gls{centrale inertielle} :
	\subitem- 	Accéléromètre 3-axes. 
	\subitem-	Gyroscope 3-axes.
	\item[•] Charge, lecture et config. par USB-C.
\end{itemize}

% ---- TACHES ----
\subsection{Tâches à réaliser}
Développement et intégration d’un PCB avec capteurs et logging sur carte SD dans un boitier compact.
\begin{itemize}
	\item[•] Développement schématique 
	\subitem- Fonctionnement MCU.
	\subitem-	Périphériques de mesures et de sauvegarde / Bus de communication.
	\subitem-	Gestion batterie 
	\item[•]	Routage pour intégration dans boitier résistant aux chocs.
	\item[•]	Programmation mesure et sauvegarde des données.
	\subitem-	Configuration MCU.
	\subitem-	Configuration du périphérique de mesure pour \gls{imu}.
	\subitem-	Configuration du périphérique de sauvegarde (Carte SD).
	\subitem-	Configuration du périphérique de localisation \gls{gps}/\gls{gnss}.
	\subitem-	Configuration et communication avec l'interface.
	\subitem-	Communication et traitement des données mesurées.
\end{itemize}

\subsection{Schéma de principe}
% ---- SChema de principe ----
\begin{figure}[h]
	\centering
	\includegraphics[width=0.4\linewidth]{../figures/cdc/schema_principe}
	\caption{Schéma de principe.}
	\source{Auteur}
	\label{fig:schemaprincipe}
\end{figure}

Ce système électronique de mini boîte noire pour avion, serait capable d'enregistrer des informations récentes sur les données inertielle et la position d'un vol, dans une mémoire non volatile (carte microSD). Le dispositif, abrité dans un boîtier plastique pour assurer une réception optimale des données \gls{gps} et une installation compact. Celui-ci fournirait des données via un port USB pour l'extraction des mesures sur la carte microSD ou pour configurer des paramètres tels que les intervalles de mesures. Les mesures sur les trois axes d'accélération et de vitesse angulaire seraient recueillies par défaut toutes les 500 ms et les données de position \gls{gps} toutes les 5000 ms. Des intervalles d'enregistrement plus longs sont envisageables pour optimiser la durée de vie de la carte SD, selon la taille et l'organisation des données. Le dispositif sauvegarderait les données des 15 dernières minutes de vol (ou plus) dans un fichier CSV pour traitement ultérieur, selon le principe FIFO. L'objectif principal du prototype est de privilégier la compacité pour minimiser son encombrement à bord de l'avion.


\clearpage

% ---- JALONS ----
\subsection{Planification}
\begin{figure}[h]
	\centering
	\includegraphics[width=1\linewidth]{../figures/cdc/planif}
	\caption{Planification, Diagramme de gantt.}
	\source{Auteur}
	\label{fig:planification}
\end{figure}
\begin{figure}[h]
	\centering
	\includegraphics[width=1\linewidth]{../figures/cdc/planif_theorique}
	\caption{Planification théorique.}
	\source{Auteur}
	\label{fig:planiftheorique}
\end{figure}


\subsection{Livrable}
\begin{itemize}
	\item[•] Les fichiers sources de CAO électronique des PCB réalisés
	\item[•] Tout le nécessaire à fabriquer un exemplaire hardware de chaque :
	\item[•] fichiers de fabrication (GERBER) / liste de pièces avec références pour commande / implantation
	\item[•] Prototype fonctionnel
	\item[•] Modifications / dessins mécaniques, etc
	\item[•] Les fichiers sources de programmation microcontrôleur (.c  / .h)
	\item[•] Tout le nécessaire pour programmer les microcontrôleurs (logiciel ou fichier .hex)
	\item[•] Un calcul / estimation des coûts
	\item[•] Un rapport contenant les calculs - dimensionnement de composants - structogramme, etc.
\end{itemize}

\clearpage


% ------------------- PRE-ETUDE ---------------------------------
\section{Pré-étude} \label{sec:Pre-Etude}

\subsection{Fonctionnement du système} \label{ssec:Fonctionnement}
Le microcontrôleur interagît avec 4 périphériques principaux : Avec le \gls{GNSS}, il partage une communication qui lui permet d'obtenir les informations de localisation par le biais de plusieurs systèmes de satellites. Il y a ensuite, la centrale inertielle qui lui donne accès de une multitude de mesures sur 9 axes, or, ici les mesures gyroscopiques et d'accélération sont exploitées. La carte SD, permet quant-à-elle, de stocker toutes ces données pour avoir minimum les information des 15 dernières minutes de vol. Le dernier périphériques principale \gls{FTDI}, permet d'avoir une interface avec un ordinateur via connexion USB-C.

\subsection{Schéma bloc} \label{ssec:Schema-bloc}

\begin{figure}[h]
	\centering
	\includegraphics[width=1\textwidth]{../figures/cdc/blocs_grossiers_no_antenna.jpg}
	\caption{Schéma bloc}
	\label{fig:schbloc}
	\source{Auteur}
\end{figure}
% ---- Description des blocs ----
\subsection{Description des blocs} \label{ssec:Desc-blocs}

\begin{table}[h]
	\resizebox{\columnwidth}{!}{%
		\begin{tabular}{|l|l|}
			\hline
			Bloc         & Description                                                                         \\ \hline
			\Gls{gnss}.  & Récepteur \Gls{rf} avec antenne interne/externe et communication UART.              \\ \hline
			\Gls{mcu}.   & Microcontrôleur PIC32, intelligence du système, basse consommation.                 \\ \hline
			\Gls{imu}.   & Centrale inertielle, accélération, gyroscope...                                     \\ \hline
			Carte SD     & Stockage des données de vol.                                                        \\ \hline
			\Gls{FTDI}.  & Convertis la communication USB en série.                                            \\ \hline
			Régulateurs. & Le régulateur de charge gère la charge de l'accu. et un régulateur 3.3V le suit.    \\ \hline
			Batterie.    & La batterie est un accu que l'on peut charger par USB et permet une bonne autonomie. \\ \hline
		\end{tabular}%
	}
\end{table}

\clearpage
\raggedbottom
\subsection{Choix des composants et technologies} \label{sssec:ComposantsTech}
L'objectif de la pré-étude consiste en grande partie à sélectionner méthodiquement les technologies et les composants du projet. Cette partie du travail est essentielle et critique.

\subsubsection{Microcontrôleur}
Le microcontrôleur nécessite au minimum les périphériques suivants : \\
\fbox{2 UART} \fbox{1 SPI} \fbox{1 I2C} \\
Il est préférable que le \gls{mcu} dispose de différentes configurations de gestion de puissance, notamment des modes d'économie d'énergie, afin d'avoir une maîtrise de la consommation et de permettre une meilleure autonomie. Enfin, le standard de l'école veut que les familles de microcontrôleurs PIC32 (Microchip) sont préférées.

\begin{figure}[h]
	\centering
	\includegraphics[width=0.65\linewidth]{../figures/pre_etude/familles_pic32}
	\caption{Familles PIC32}
	\label{fig:famillespic32}
	\source{\href{https://www.microchip.com/en-us/products/microcontrollers-and-microprocessors/32-bit-mcus/pic32-32-bit-mcus}{https://www.microchip.com/en-us/products/microcontrollers-and-microprocessors/32-bit-mcus/pic32-32-bit-mcus}}
\end{figure}

Sur la figure \ref{fig:famillespic32} le \gls{mcu} sélectionné appartient à la gamme MX (Baseline performance-mémoire) et à la famille XLP qui offre notamment la fonctionnalité "Ultra low power" qui est celle qui nous intéresse.

\begin{figure}[h]
	\centering
	\includegraphics[width=0.7\linewidth]{../figures/pre_etude/Carac_PIC32}
	\caption{Caractéristiques du PIC32 choisi}
	\source{Auteur}
	\label{fig:caracpic32}
\end{figure}

\clearpage
\subsubsection{Centrale inertielle} 
Pour la centrale inertielle, il existe un composant avec lequel j'ai déjà acquis une certaine expérience et eu l'occasion d'utiliser et de créer des librairies pour le firmware en C. Celui-ci est performant et très utilisé dans l'industrie. Il est hautement configurable et a le grand avantage de calculer déjà une fusion de capteurs ainsi qu'une compensation de la dérive par les mesures de température. Cela permet notamment d'accéder à des données plus poussées, telles que les quaternions et les angles d'Euler. Il s'agit du \fbox{BNO055} de BOSCH.
 
\begin{center}
	\underline{Caractéristiques importantes :} \\
	\begin{tabular}{l l l l}
		Résolution gyroscope & : & 16 & [bits] \\
		Résolution accéléromètre & : & 14 & [bits] \\
		Résolution magnétomètre & : & $\sim$0.3 & [$\mu$T] \\
		$I_{DD}$ & : & 12.3 & [mA] \\
		Dérive de température & : & $\pm$ 0.03 & [\%/K] \\ 
		Dérive accéléromètre & : & 0.2 & [\%/V] \\
		Dérive gyroscope & : & <0.4 & [\%/V]
	\end{tabular} \\
\end{center}

Afin de simplifier l'implémentation de ce composant dans le projet, sachant qu'il s'agit d'un boîtier de composant difficile à souder ou à mettre au four, il est possible d'utiliser la carte miniature d'Adafruit, cette carte comprend tous les composants passifs requis. Cela facilite ainsi son montage sur le PCB.

\begin{figure}[h]
	\centering
	\includegraphics[width=0.4\linewidth]{../figures/pre_etude/BNO055_Adafruit}
	\caption{Carte d'extension, centrale inertielle}
	\source{\href{https://www.digikey.ch/en/products/detail/adafruit-industries-llc/4646/12609996?s=N4IgTCBcDaIEYDsD2AGArGgBAQwCbYDMAnAVwEsAXEAXQF8g}{Digikey, 4646}}
	\label{fig:bno055adafruit}
\end{figure}

\begin{center}
	\underline{Données disponibles :}
	\begin{table}[h]
		\centering
		\begin{tabular}{|ll|}
		\hline
		Température & \\
		\hline
		Vecteur gravité & XYZ \\
		\hline
		Orientation compensées, quaternion & WXYZ \\
		\hline
		Orientation compensée, angle de Euler & HPR \\
		\hline
		Données gyroscopiques & XYZ \\
		\hline
		 Intensité du champ magnétique & XYZ \\
		\hline
		Accélération & XYZ \\
		\hline
	\end{tabular}
	\caption{Liste des données accessibles}
	\end{table}
\end{center}

\clearpage

\subsubsection{GPS / GNSS}
Pour le \gls{GPS}/\gls{GNSS}, différents critères entrent en jeu dans le cadre de ce projet : le prix, la facilité d'implémentation, la complexité (par complexité, nous entendons le nombre de fonctionnalités), la consommation et la performance.

Il existe un très grand nombre de récepteurs \gls{rf} pour la navigation. Parmi les plus utilisés dans l'industrie, dont l'implémentation est la plus simple et la documentation la plus complète, il y a plusieurs gammes chez le fabricant \fbox{ublox}.
Deux composants ont principalement été pris en considération :
Le \textbf{CAM-M8C-0} (BeiDou, GLONASS, GNSS, GPS, QZSS) avec une antenne omnidirectionnelle interne au composant et différents modes de puissance, ainsi que le \textbf{MAX-M10M-00B} (BeiDou, Galileo, GLONASS, GNSS, GPS) sans antenne interne mais avec une consommation de base plus faible.

\begin{figure}[h]
	\centering
	\includegraphics[width=0.6\linewidth]{../figures/pre_etude/img_gnss}
	\caption{Illustration des deux GNSS}
	\source{\href{https://www.digikey.ch/fr/products/detail/u-blox/CAM-M8C-0/6150647?s=N4IgTCBcDaIMIEECyBaJAOOKAMIC6AvkA}{Digikey}}
	\label{fig:imggnss}
\end{figure}


\begin{figure}[h]
	\centering
	\includegraphics[width=0.65\linewidth]{../figures/pre_etude/Comp_GNSS}
	\caption{Comparaison GNSS}
	\source{Auteur}
	\label{fig:compgnss}
\end{figure}

Malgré les différents avantages que présente le MAX-M10M-00B sur la figure \ref{fig:compgnss}, le choix s'est porté sur CAM-M8C-0 grâce à sa facilitée d'implémentation et à la garantie du fabricant sur son antenne omnidirectionnelle de qualité.

\clearpage

\subsubsection{Carte SD} 
Les pilotes de carte SD disponibles dans \gls{harmony} ne permettent pas une gestion des capacités de stockage trop importantes. Cela signifie, que nous ne pouvons pas avoir des cartes SD de trop grandes capacités, c'est pour cela que nous allons baser nos dimensionnements sur une carte de \textbf{256MB}. 

\begin{figure}[h]
	\centering
	\includegraphics[width=0.2\linewidth]{../figures/pre_etude/CarteSD_Illustration}
	\caption{Illustration carte SD}
	\source{\href{https://www.oempcworld.com/OEMPCworld-com/017045.html}{https://www.oempcworld.com/OEMPCworld-com/017045.html}}
	\label{fig:cartesdillustration}
\end{figure}


\paragraph{Estimation de la capacité} Admettons les paramètres suivants où $S$ représente une taille de données et $T$ un temps :

\begin{tabular}{lrl}
	$S_{SD}$ & $256$ & $[KB]$ \\
	$S_{gyro}$ & $16$ & $[Bytes]$ \\
	$S_{accel}$ & $16$ & $[Bytes]$ \\
	$S_{gnss}$ & $\sim100$ & $[Bytes]$ \\
	$T_{inertiel}$ & $0.5$ & $[s]$ \\
	$T_{gnss}$ & $5$ & $[s]$ \\
	$T_{mesMin}$ & $900$ & $[s]$ \\
\end{tabular}

Nous pouvons déduire la taille mémoire que prendra 5 secondes d'enregistrement avec les paramètres par défaut du système : 
\begin{equation*}
	S_{single} = \frac{T_{gnss}}{T_{inertiel}}(S_{gyro}+S_{accel}) + S_{gnss} = \frac{5}{0.5}(16+16) + 100 = 420 \; [Bytes]
\end{equation*}

Nous pouvons enfin calculer à partir de cela, la taille mémoire que prendra 15 minutes d'enregistrement : 

\begin{equation*}
	S_{mesures} = \frac{S_{single}}{T_{gnss}} * T_{mesMin} = \frac{420}{5} * 900 = 75'600 \; [Bytes] = 75.6 \; [KB]
\end{equation*}

Nous pouvons déduire avec cette estimation qu'une carte SD de 256MB est largement suffisante et permet de mesurer jusqu'à un temps calculable de cette façon :

\begin{equation*}
	T_{mesures} = \frac{S_{SD}*T_{gnss}}{S_{single}} = \frac{256'000*5}{420} = \sim508 \; Minutes = \sim8.5 \; Heures
\end{equation*}

Nous estimons donc, que les données de vol des 8.5 dernières heures sont enregistrées dans la carte SD.

\clearpage

\subsubsection{Batterie, charge et régulation} 
Afin de dimensionner une batterie pour le projet, il faut considérer les différentes consommation :

\begin{center}
	\underline{Liste des consommations principales} \\
	\begin{table}[h]
		\centering
		\begin{tabular}{lrll}
			Microcontrôleur & 24 & [mA] & Typ. \\
			Carte-SD & ~100 & [mA] & Max. \\
			Carte-SD & ~60 & [mA] & Moyenne \\
			\gls{imu} & 12.3 & [mA] & Typ. \\
			\gls{GNSS} & 71 & [mA] & Max. \\
			\gls{GNSS} & 29 & [mA] & Typ. \\
			\hline
			Totale max & \underline{207.3} & [mA] & Max. \\
			Totale moyennes & \underline{125.3} & [mA] & Moyenne \\
			\hline
		\end{tabular}
		\caption{Tableau des consommations de courant}
		\label{tab:consommateur}
	\end{table}
\end{center}

En examinant les consommations typiques et moyennes du tableau \ref{tab:consommateur} à régime constant, et en visant une autonomie de \fbox{\textbf{10 heures}}, nous aurions besoin d'une batterie d'une capacité de \fbox{\textbf{1253 mAh}}.

\paragraph{Technologies} Concernant la technologie de la batterie, notre objectif, dans un souci d'ergonomie, est de permettre son chargement via un connecteur USB. La technologie offrant actuellement la meilleure densité d'énergie est le \fbox{Lithium Ion}\footnote{\href{https://www.epectec.com/batteries/cell-comparison.html}{https://www.epectec.com/batteries/cell-comparison.html}}. Elle présente cependant des inconvénients, énumérés dans le tableau \ref{tab:inconvlion}.

\begin{table}[h]
	\centering
	\begin{tabular}{l|l}
		Avantages &  Inconvénient\\
		\hline
		Haute densité d'énergie & Risque d'éclatement \\
		Poids léger & Risque d'enflammement avec l'eau \\ 
		Haute durée de vie & Sensible a la température \\
		Charge rapide & Décharge complète altérante \\
		\hline
	\end{tabular}
		\caption{Tableau avantages/inconvénient LI-ION}
		\label{tab:inconvlion}
\end{table}

Les inconvénients listés dans le tableau \ref{tab:inconvlion} posent un risque pour l'intégrité des données de la carte-SD. C'est pourquoi il est essentiel de bien dimensionner le boîtier ainsi que l'intégration du circuit et de la batterie pour pallier ces dangers. Malgré les risques du Li-Ion, il s'agit d'une technologie très utilisée, y compris dans des domaines sensibles.

\clearpage

\paragraph{Choix de la batterie} La batterie doit avoir une capacité d'au moins \textbf{1260 mAh} et être suffisamment compacte pour être intégrée dans un boîtier de petite taille. Une batterie répond à ces critères : il s'agit de la \fbox{\textbf{PICPAL36}} de chez Farnell\footnote{Distributeur de composants électroniques.}. Les dimensions et caractéristiques de cette batterie sont visibles sur les figures \ref{fig:batt} et \ref{fig:caracbatt}.

\begin{figure}[h]
	\centering
	\includegraphics[width=0.55\linewidth]{../figures/pre_etude/batt}
	\caption{Dimension de la batterie}
	\label{fig:batt}
	\source{Auteur}
\end{figure}

\begin{figure}[h]
	\centering
	\includegraphics[width=0.7\linewidth]{../figures/pre_etude/Carac_Batt}
	\caption{Caractèristiques de la batterie}
	\label{fig:caracbatt}
\end{figure}

\paragraph{Régulateur de charge} Le choix du régulateur de charge s'est orienté vers un composant déjà utilisé par l'école, dont les caractéristiques et le montage sont bien connus. C'est un composant performant et fréquemment employé. De plus, il est disponible dans le stock de composants de l'école. Il s'agit du \fbox{\textbf{MCP73871T-2CCI/ML}}. Il est configurable et permet la charge des batteries Li-ion et Li-po.

\begin{figure}[h]
	\centering
	\includegraphics[width=0.2\linewidth]{../figures/pre_etude/MCP}
	\caption{MCP73871T-2CCI/ML}
	\source{\href{https://www.digikey.ch/fr/products/detail/microchip-technology/MCP73871T-2CCI-ML/7065594}{Digikey MCP73871}}
	\label{fig:mcp}
\end{figure}


\clearpage

\subsection{Systèmes d'économies d'énergie} \label{ssec:Low-power}
Dans le but de maximiser le temps de logging, des mécanismes d'économies d'énergie doivent être mis en place. 

\subsection{Estimation des coûts} \label{ssec:Estimation-Couts}


% ---------- DÉVELOPPEMENT SCHÉMATIQUE --------------------------
\section{Développement du schéma électronique} \label{sec:Dev-Schematique}
Dans cette section, nous décrirons la phase principale du développement ainsi que la démarche suivie pour élaborer le schéma électronique du projet.

\subsection{Blocs développés} \label{ssec:Dev-blocs}
Pour faciliter le développement et la lecture du schéma, il est judicieux de diviser le système en plusieurs blocs. Une structure a ainsi été définie, divisant le circuit en trois blocs principaux : \hyperref[ssec:Dev-MCU]{\textbf{Microcontrôleur \ref{ssec:Dev-MCU}}} (Intelligence du système, connexion du programmeur et LED de vie.), \hyperref[ssec:Dev-Devices]{\textbf{Périphériques \ref{ssec:Dev-Devices}}} (\gls{GNSS}, \gls{imu}, \gls{FTDI}, connecteur USB, Carte SD.) et \hyperref[ssec:Dev-Power]{\textbf{Puissance \ref{ssec:Dev-Power}}} (Connecteur batterie, gestion de charge, régulateurs de tension et système ON/OFF.).

Nous pouvons sur la figure \ref{fig:blocs} observer les différentes interaction entre les blocs, elles sont par la suite décrites dans le tableau \ref{tab:descrConnexion}.

\begin{figure}[h]
	\centering
	\includegraphics[width=1\linewidth]{../figures/etude/sch/BLOCS}
	\caption{Blocs du système}
	\source{Auteur}
	\label{fig:blocs}
\end{figure}

\begin{center}
	\underline{Tableau des interaction entre les blocs}
	\begin{table}[h]
		\centering
		\resizebox{\columnwidth}{!}{%
			\begin{tabular}{l|l}
				Connexion/s & Description \\
				\hline
				INT\_IMU & Interruption de la centrale inertielle, informe le \gls{mcu} et peut allumer le système. \\ 
				RST\_IMU & Permet de réinitialiser la \gls{imu}. \\
				PWR\_HOLD & Le \gls{mcu} peut se maintenir alimenté par cette connexion. \\
				BAT\_STAT & Fournit le statut de la batterie au \gls{mcu}. \\
				Button\_MF & Fournit le niveau logique du bouton au \gls{mcu}. \\
				BAT\_READ & Le \gls{mcu} peut activer la lecture de la tension de la batterie. \\
				BAT\_LVL & Lecture analogique de la tension de batterie. \\
				SDA1, SCL1 & Communication I2C avec l'\gls{imu}. \\
				SDA2, SCL2 & Communication I2C optionnelle avec le \gls{GNSS}. \\
				U1TX/RX... & Communication UART avec le \gls{FTDI} pour l'USB. \\
				U2TX/RX & Communication UART avec le \gls{GNSS}. \\
				SCK, MOSI... & Communication SPI avec la carte SD. \\ 
				R1,2,3,4 & Résistances de PULL-UP pour communication I2C. \\
			\end{tabular}
		}
		\caption{Description des connexions}
		\label{tab:descrConnexion}
	\end{table}
\end{center}

\clearpage

\subsection{Microcontrôleur} \label{ssec:Dev-MCU}

\subsubsection{Connexion} 
Pour utiliser le microcontrôleur, il est nécessaire de définir ses entrées/sorties en se référant à son \gls{datasheet}. Ce dernier permet de connaître les connexions dédiées à certains bus de communication ou à des entrées analogiques. 

\begin{figure}[h]
	\centering
	\begin{subfigure}[b]{0.7\textwidth}
		\centering
		\includegraphics[width=1\linewidth]{../figures/etude/sch/MCU}
		\caption{Connexions du microcontrôleur}
		\label{fig:mcu}
	\end{subfigure}
	\hfill
	\begin{subfigure}[b]{0.25\textwidth}
		\centering
		\includegraphics[width=1\linewidth]{../figures/etude/sch/MCU-HARMONY}
		\caption{Config. \gls{harmony}.}
		\label{fig:mcu-harmony}
	\end{subfigure}
	\hfill
	\caption{Configuration des PINs du microcontrôleur}
	\source{Auteur}
	\label{fig:sch-connMcu}
\end{figure}

Pour valider les connexions de la figure \ref{fig:mcu}, une vérification a été effectuée avec le configurateur graphique \gls{harmony}, illustrée par la figure \ref{fig:mcu-harmony}. Cette démarche a confirmé la possibilité de dédier certaines fonctions à des PINs.

\subsubsection{Programmateur et reset} 

\begin{figure}[h]
	\centering
	\includegraphics[width=0.7\linewidth]{../figures/etude/sch/Prog-Reset}
	\caption{Schéma programmateur et reset}
	\source{Auteur}
	\label{fig:prog-reset}
\end{figure}

Sur la figure \ref{fig:prog-reset}, nous pouvons observer le connecteur de programmation \textit{P3}. La connexion \textbf{MCLR} (Master Clear), qui permet de réinitialiser le \gls{mcu} lors de sa programmation, est suivie d'un circuit de protection du \gls{mcu} et d'un bouton pour permettre une réinitialisation manuelle.

\clearpage

\subsubsection{LED de vie} 

\begin{figure}[h]
	\centering
	\includegraphics[width=.8\linewidth]{../figures/etude/DIM-LED}
	\caption{Données de la LED}
	\source{\gls{datasheet} \href{https://docs.broadcom.com/doc/ASMB-KTF0-0A306-DS100}{ASMB-KTF0-0A306}}
	\label{fig:dim-led}
\end{figure}

\begin{figure}[h]
	\centering
	\includegraphics[width=0.4\linewidth]{../figures/etude/sch/LED-Vie}
	\caption{Schéma de la LED de vie}
	\source{Auteur}
	\label{fig:led-vie}
\end{figure}

Les résistances de la figure \ref{fig:led-vie} ont été dimensionnées en respectant les caractéristiques des LEDs pour chacune des couleurs (voir figure \ref{fig:dim-led}). Ces dernières éclairent à 80\% de leur luminosité nominale dans un souci d'économie d'énergie, leur utilité étant réduite à ce stade de prototypage.

\begin{center}
	\underline{Exemple dimensionnement résistance LED bleue}
	\begin{equation*}
		R_{blue} = \frac{(Vcc - V_{blue})}{I_b} = \frac{(3.3 - 2.85)}{12*10^-3} = 37.5 \Omega
	\end{equation*}
\end{center}

\clearpage

\subsection{Périphériques} \label{ssec:Dev-Devices}

\subsection{Puissance} \label{ssec:Dev-Power}

\subsection{Dimensionnements} \label{ssec:Dev-Dimensionnements}

\subsubsection{Bus de communications} \label{sssec:Dev-BusComm}

\subsubsection{Interface} \label{sssec:Interface}

\subsubsection{Périphériques} \label{sssec:Peripheriques}

\subsubsection{Chargeur de batterie} \label{sssec:Chargeur-bat}

\subsubsection{Adaptation mécanique} \label{sssec:Adaptation-mech}

\subsection{Synthèse et perspectives de l'étude} \label{ssec:Synth-etude}

% ---------- DÉVELOPPEMENT PCB ----------------------------------
\section{Développement du PCB} \label{sec:Dev-PCB}

\subsection{Bill of materials} \label{ssec:BOM}

\subsection{Mécanique du projet} \label{ssec:mechProjet}

\subsection{Placement des composants} \label{ssec:placementComp}

\subsection{Mécanique du PCB} \label{ssec:Mech-PCB}

\subsection{Routage} \label{ssec:routage}



% ---------- DÉVELOPPEMENT SOFTWARE ----------------------------- 
\section{Développement firmware} \label{sec:Dev-firmware}


% ---------- MESURE PREUVE DE CONCEPT --------------------------- 
\clearpage

\section{Validation du design} \label{sec:Validation-design}

\subsection{Liste de matériel} \label{ssec:Liste-materiel}

\subsection{Consommations}
\subsubsection{Méthode de mesure}
\subsubsection{Mesures}


\subsection{Communication I2C} \label{ssec:Comm-I2C}
\subsubsection{Méthode de mesure}
\subsubsection{Mesures}

\subsection{Communication UART} \label{ssec:Comm-UART}
\subsubsection{Méthode de mesure}
\subsubsection{Mesures}

\subsection{Communication SPI, carte SD} \label{ssec:Comm-SPI}
\subsubsection{Méthode de mesure}
\subsubsection{Mesures}

\section{Caractéristiques du produit fini} \label{sec:Carac-finis}

% ---- Bibliographie ----
%\input{bibliography}

\clearpage

\section{Conclusion}

Ce rapport a présenté en détail le développement d'une mini boîte noire destinée à collecter et stocker des données de vol d'aéronefs en utilisant une centrale inertielle et un système de positionnement \gls{gps}/\gls{gnss}. Le projet a débuté par une phase de pré-étude visant à comprendre le fonctionnement du système et à choisir les composants et les technologies appropriés, notamment le microcontrôleur, la centrale inertielle, le \gls{gnss}, et la carte SD. Ensuite, le développement du schéma électronique a été abordé, détaillant la connectivité du microcontrôleur, des périphériques tels que la carte SD, la centrale inertielle, et le module USB-FTDI, ainsi que les alimentations et le système de charge de la batterie. La conception du circuit-imprimé a été décrite, mettant l'accent sur les aspects mécaniques, le placement des composants, l'assemblage, et la liste des matériaux.

La phase de développement du firmware a été exposée, couvrant les protocoles du \gls{gnss}, la configuration des périphériques, l'application principale gérant les commandes USB-UART et l'état de logging, ainsi que la gestion des données de la carte SD, de la centrale inertielle, et des communications série avec le \gls{FTDI}. L'interface utilisateur de l'application a été brièvement décrite, ainsi que les critères de validation du design, incluant la consommation d'énergie, les bus de communication I2C-\gls{imu}, UART-\gls{gnss}, UART-\gls{FTDI} et SPI pour la carte SD.

Ce projet témoigne de l'importance des boîtes noires dans l'aviation,  représentant un exemple concret de l'application de technologies avancées pour améliorer la sécurité aérienne et la compréhension des incidents aéronautiques. L'objectif de stocker des données de mesures a été atteint avec succès, et ce projet ouvre des perspectives passionnantes pour le développement futur de systèmes similaires.

Je tiens a adresser mes sincères remerciements à M. J. J. Moreno pour son expertise et son aide précieuse tout au long du développement de ce projet. Mes remerciements s'étendent également à l'AMPA pour avoir mandaté ce projet passionnant et pour la confiance qu'ils m'ont accordée.


\newpage
\nocite{*}
\section{Bibliographie}
\bibliography{Biblio_TDD} 
\bibliographystyle{ieeetr}


% ANNEXES
\clearpage
\section{Annexes}

\includepdf[pages=1,landscape=true, scale=.8,pagecommand={\subsection{Schéma}\label{pdf:schéma}},linktodoc=true]{../annexes/Schematic-Prints.pdf}
\includepdf[pages=2-, landscape=true, scale=.8,pagecommand={},linktodoc=true]{../annexes/Schematic-Prints.pdf}

\includepdf[pages=1,  landscape=true, scale=.8,pagecommand={\subsection{Dessin du boitier}\label{pdf:dessin-boitier}},linktodoc=true]{../annexes/Dessin-boitier.pdf}
\includepdf[pages=2-,  landscape=true, scale=.8,pagecommand={},linktodoc=true]{../annexes/Dessin-boitier.pdf}

\includepdf[pages=1,  scale=.8,pagecommand={\subsection{Fichiers PCB}\label{pdf:pcb-files}},linktodoc=true]{../annexes/PCB Top-Bottom Assy Dwgs.pdf}
\includepdf[pages=2-,scale=.8,pagecommand={},linktodoc=true]{../annexes/PCB Top-Bottom Assy Dwgs.pdf}

\includepdf[pages=1,  scale=.8,pagecommand={\subsection{Code du firmware C}\label{pdf:code-firmware}},linktodoc=true]{../annexes/code/_Code_complet.pdf}
\includepdf[pages=2-,scale=.8,pagecommand={},linktodoc=true]{../annexes/code/_Code_complet.pdf}

\includepdf[pages=1,  scale=.8,pagecommand={\subsection{Code python du script visualisation CSV}\label{pdf:python-import-csv}},linktodoc=true]{../annexes/Script-import-csv.pdf}
\includepdf[pages=2-,scale=.8,pagecommand={},linktodoc=true]{../annexes/Script-import-csv.pdf}

\includepdf[pages=1,  scale=.8,pagecommand={\subsection{Code python de l'application  BBX-Connect}\label{pdf:app-python}},linktodoc=true]{../annexes/application_python.pdf}
\includepdf[pages=2-,scale=.8,pagecommand={},linktodoc=true]{../annexes/application_python.pdf}

\includepdf[pages=1,  scale=.8,pagecommand={\subsection{Fichier de modification}\label{pdf:FichierDeModif}},linktodoc=true]{../annexes/1942B_MiniBoiteNoire-MOD-v1.pdf}

\includepdf[pages=1,  scale=.8,pagecommand={\subsection{Journal de travail}\label{pdf:journal}},linktodoc=true]{../annexes/journal.pdf}

\includepdf[pages=1,  scale=.8,pagecommand={\subsection{Historique Git}\label{pdf:gitHistory}},linktodoc=true]{../annexes/historique-git.pdf}
\includepdf[pages=2-,scale=.8,pagecommand={},linktodoc=true]{../annexes/historique-git.pdf}


\end{document}
